
%% bare_jrnl.tex
%% V1.4b
%% 2015/08/26
%% by Michael Shell
%% see http://www.michaelshell.org/
%% for current contact information.
%%
%% This is a skeleton file demonstrating the use of IEEEtran.cls
%% (requires IEEEtran.cls version 1.8b or later) with an IEEE
%% journal paper.
%%
%% Support sites:
%% http://www.michaelshell.org/tex/ieeetran/
%% http://www.ctan.org/pkg/ieeetran
%% and
%% http://www.ieee.org/

%%*************************************************************************
%% Legal Notice:
%% This code is offered as-is without any warranty either expressed or
%% implied; without even the implied warranty of MERCHANTABILITY or
%% FITNESS FOR A PARTICULAR PURPOSE! 
%% User assumes all risk.
%% In no event shall the IEEE or any contributor to this code be liable for
%% any damages or losses, including, but not limited to, incidental,
%% consequential, or any other damages, resulting from the use or misuse
%% of any information contained here.
%%
%% All comments are the opinions of their respective authors and are not
%% necessarily endorsed by the IEEE.
%%
%% This work is distributed under the LaTeX Project Public License (LPPL)
%% ( http://www.latex-project.org/ ) version 1.3, and may be freely used,
%% distributed and modified. A copy of the LPPL, version 1.3, is included
%% in the base LaTeX documentation of all distributions of LaTeX released
%% 2003/12/01 or later.
%% Retain all contribution notices and credits.
%% ** Modified files should be clearly indicated as such, including  **
%% ** renaming them and changing author support contact information. **
%%*************************************************************************


% *** Authors should verify (and, if needed, correct) their LaTeX system  ***
% *** with the testflow diagnostic prior to trusting their LaTeX platform ***
% *** with production work. The IEEE's font choices and paper sizes can   ***
% *** trigger bugs that do not appear when using other class files.       ***                          ***
% The testflow support page is at:
% http://www.michaelshell.org/tex/testflow/



\documentclass[journal]{IEEEtran}
%
% If IEEEtran.cls has not been installed into the LaTeX system files,
% manually specify the path to it like:
% \documentclass[journal]{../sty/IEEEtran}





% Some very useful LaTeX packages include:
% (uncomment the ones you want to load)


% *** MISC UTILITY PACKAGES ***
%
%\usepackage{ifpdf}
% Heiko Oberdiek's ifpdf.sty is very useful if you need conditional
% compilation based on whether the output is pdf or dvi.
% usage:
% \ifpdf
%   % pdf code
% \else
%   % dvi code
% \fi
% The latest version of ifpdf.sty can be obtained from:
% http://www.ctan.org/pkg/ifpdf
% Also, note that IEEEtran.cls V1.7 and later provides a builtin
% \ifCLASSINFOpdf conditional that works the same way.
% When switching from latex to pdflatex and vice-versa, the compiler may
% have to be run twice to clear warning/error messages.






% *** CITATION PACKAGES ***
%
\usepackage{cite}
% cite.sty was written by Donald Arseneau
% V1.6 and later of IEEEtran pre-defines the format of the cite.sty package
% \cite{} output to follow that of the IEEE. Loading the cite package will
% result in citation numbers being automatically sorted and properly
% "compressed/ranged". e.g., [1], [9], [2], [7], [5], [6] without using
% cite.sty will become [1], [2], [5]--[7], [9] using cite.sty. cite.sty's
% \cite will automatically add leading space, if needed. Use cite.sty's
% noadjust option (cite.sty V3.8 and later) if you want to turn this off
% such as if a citation ever needs to be enclosed in parenthesis.
% cite.sty is already installed on most LaTeX systems. Be sure and use
% version 5.0 (2009-03-20) and later if using hyperref.sty.
% The latest version can be obtained at:
% http://www.ctan.org/pkg/cite
% The documentation is contained in the cite.sty file itself.

\newcommand{\tabincell}[2]{\begin{tabular}{@{}#1@{}}#2\end{tabular}}
\usepackage{textcomp}
\usepackage{algorithm}
\usepackage{subfigure}
\usepackage{booktabs}

% *** GRAPHICS RELATED PACKAGES ***
%
\ifCLASSINFOpdf
  \usepackage[pdftex]{graphicx}
  % declare the path(s) where your graphic files are
  % \graphicspath{{../pdf/}{../jpeg/}}
  % and their extensions so you won't have to specify these with
  % every instance of \includegraphics
  % \DeclareGraphicsExtensions{.pdf,.jpeg,.png}
\else
  % or other class option (dvipsone, dvipdf, if not using dvips). graphicx
  % will default to the driver specified in the system graphics.cfg if no
  % driver is specified.
  % \usepackage[dvips]{graphicx}
  % declare the path(s) where your graphic files are
  % \graphicspath{{../eps/}}
  % and their extensions so you won't have to specify these with
  % every instance of \includegraphics
  % \DeclareGraphicsExtensions{.eps}
\fi
% graphicx was written by David Carlisle and Sebastian Rahtz. It is
% required if you want graphics, photos, etc. graphicx.sty is already
% installed on most LaTeX systems. The latest version and documentation
% can be obtained at: 
% http://www.ctan.org/pkg/graphicx
% Another good source of documentation is "Using Imported Graphics in
% LaTeX2e" by Keith Reckdahl which can be found at:
% http://www.ctan.org/pkg/epslatex
%
% latex, and pdflatex in dvi mode, support graphics in encapsulated
% postscript (.eps) format. pdflatex in pdf mode supports graphics
% in .pdf, .jpeg, .png and .mps (metapost) formats. Users should ensure
% that all non-photo figures use a vector format (.eps, .pdf, .mps) and
% not a bitmapped formats (.jpeg, .png). The IEEE frowns on bitmapped formats
% which can result in "jaggedy"/blurry rendering of lines and letters as
% well as large increases in file sizes.
%
% You can find documentation about the pdfTeX application at:
% http://www.tug.org/applications/pdftex





% *** MATH PACKAGES ***
%
\usepackage{amsmath}
% A popular package from the American Mathematical Society that provides
% many useful and powerful commands for dealing with mathematics.
%
% Note that the amsmath package sets \interdisplaylinepenalty to 10000
% thus preventing page breaks from occurring within multiline equations. Use:
%\interdisplaylinepenalty=2500
% after loading amsmath to restore such page breaks as IEEEtran.cls normally
% does. amsmath.sty is already installed on most LaTeX systems. The latest
% version and documentation can be obtained at:
% http://www.ctan.org/pkg/amsmath





% *** SPECIALIZED LIST PACKAGES ***
%
\usepackage{algorithmic}
% algorithmic.sty was written by Peter Williams and Rogerio Brito.
% This package provides an algorithmic environment fo describing algorithms.
% You can use the algorithmic environment in-text or within a figure
% environment to provide for a floating algorithm. Do NOT use the algorithm
% floating environment provided by algorithm.sty (by the same authors) or
% algorithm2e.sty (by Christophe Fiorio) as the IEEE does not use dedicated
% algorithm float types and packages that provide these will not provide
% correct IEEE style captions. The latest version and documentation of
% algorithmic.sty can be obtained at:
% http://www.ctan.org/pkg/algorithms
% Also of interest may be the (relatively newer and more customizable)
% algorithmicx.sty package by Szasz Janos:
% http://www.ctan.org/pkg/algorithmicx




% *** ALIGNMENT PACKAGES ***
%
\usepackage{array}
% Frank Mittelbach's and David Carlisle's array.sty patches and improves
% the standard LaTeX2e array and tabular environments to provide better
% appearance and additional user controls. As the default LaTeX2e table
% generation code is lacking to the point of almost being broken with
% respect to the quality of the end results, all users are strongly
% advised to use an enhanced (at the very least that provided by array.sty)
% set of table tools. array.sty is already installed on most systems. The
% latest version and documentation can be obtained at:
% http://www.ctan.org/pkg/array


% IEEEtran contains the IEEEeqnarray family of commands that can be used to
% generate multiline equations as well as matrices, tables, etc., of high
% quality.




% *** SUBFIGURE PACKAGES ***
%\ifCLASSOPTIONcompsoc
%  \usepackage[caption=false,font=normalsize,labelfont=sf,textfont=sf]{subfig}
%\else
%  \usepackage[caption=false,font=footnotesize]{subfig}
%\fi
% subfig.sty, written by Steven Douglas Cochran, is the modern replacement
% for subfigure.sty, the latter of which is no longer maintained and is
% incompatible with some LaTeX packages including fixltx2e. However,
% subfig.sty requires and automatically loads Axel Sommerfeldt's caption.sty
% which will override IEEEtran.cls' handling of captions and this will result
% in non-IEEE style figure/table captions. To prevent this problem, be sure
% and invoke subfig.sty's "caption=false" package option (available since
% subfig.sty version 1.3, 2005/06/28) as this is will preserve IEEEtran.cls
% handling of captions.
% Note that the Computer Society format requires a larger sans serif font
% than the serif footnote size font used in traditional IEEE formatting
% and thus the need to invoke different subfig.sty package options depending
% on whether compsoc mode has been enabled.
%
% The latest version and documentation of subfig.sty can be obtained at:
% http://www.ctan.org/pkg/subfig




% *** FLOAT PACKAGES ***
%
%\usepackage{fixltx2e}
% fixltx2e, the successor to the earlier fix2col.sty, was written by
% Frank Mittelbach and David Carlisle. This package corrects a few problems
% in the LaTeX2e kernel, the most notable of which is that in current
% LaTeX2e releases, the ordering of single and double column floats is not
% guaranteed to be preserved. Thus, an unpatched LaTeX2e can allow a
% single column figure to be placed prior to an earlier double column
% figure.
% Be aware that LaTeX2e kernels dated 2015 and later have fixltx2e.sty's
% corrections already built into the system in which case a warning will
% be issued if an attempt is made to load fixltx2e.sty as it is no longer
% needed.
% The latest version and documentation can be found at:
% http://www.ctan.org/pkg/fixltx2e


\usepackage{stfloats}
% stfloats.sty was written by Sigitas Tolusis. This package gives LaTeX2e
% the ability to do double column floats at the bottom of the page as well
% as the top. (e.g., "\begin{figure*}[!b]" is not normally possible in
% LaTeX2e). It also provides a command:
%\fnbelowfloat
% to enable the placement of footnotes below bottom floats (the standard
% LaTeX2e kernel puts them above bottom floats). This is an invasive package
% which rewrites many portions of the LaTeX2e float routines. It may not work
% with other packages that modify the LaTeX2e float routines. The latest
% version and documentation can be obtained at:
% http://www.ctan.org/pkg/stfloats
% Do not use the stfloats baselinefloat ability as the IEEE does not allow
% \baselineskip to stretch. Authors submitting work to the IEEE should note
% that the IEEE rarely uses double column equations and that authors should try
% to avoid such use. Do not be tempted to use the cuted.sty or midfloat.sty
% packages (also by Sigitas Tolusis) as the IEEE does not format its papers in
% such ways.
% Do not attempt to use stfloats with fixltx2e as they are incompatible.
% Instead, use Morten Hogholm'a dblfloatfix which combines the features
% of both fixltx2e and stfloats:
%
% \usepackage{dblfloatfix}
% The latest version can be found at:
% http://www.ctan.org/pkg/dblfloatfix




%\ifCLASSOPTIONcaptionsoff
%  \usepackage[nomarkers]{endfloat}
% \let\MYoriglatexcaption\caption
% \renewcommand{\caption}[2][\relax]{\MYoriglatexcaption[#2]{#2}}
%\fi
% endfloat.sty was written by James Darrell McCauley, Jeff Goldberg and 
% Axel Sommerfeldt. This package may be useful when used in conjunction with 
% IEEEtran.cls'  captionsoff option. Some IEEE journals/societies require that
% submissions have lists of figures/tables at the end of the paper and that
% figures/tables without any captions are placed on a page by themselves at
% the end of the document. If needed, the draftcls IEEEtran class option or
% \CLASSINPUTbaselinestretch interface can be used to increase the line
% spacing as well. Be sure and use the nomarkers option of endfloat to
% prevent endfloat from "marking" where the figures would have been placed
% in the text. The two hack lines of code above are a slight modification of
% that suggested by in the endfloat docs (section 8.4.1) to ensure that
% the full captions always appear in the list of figures/tables - even if
% the user used the short optional argument of \caption[]{}.
% IEEE papers do not typically make use of \caption[]'s optional argument,
% so this should not be an issue. A similar trick can be used to disable
% captions of packages such as subfig.sty that lack options to turn off
% the subcaptions:
% For subfig.sty:
% \let\MYorigsubfloat\subfloat
% \renewcommand{\subfloat}[2][\relax]{\MYorigsubfloat[]{#2}}
% However, the above trick will not work if both optional arguments of
% the \subfloat command are used. Furthermore, there needs to be a
% description of each subfigure *somewhere* and endfloat does not add
% subfigure captions to its list of figures. Thus, the best approach is to
% avoid the use of subfigure captions (many IEEE journals avoid them anyway)
% and instead reference/explain all the subfigures within the main caption.
% The latest version of endfloat.sty and its documentation can obtained at:
% http://www.ctan.org/pkg/endfloat
%
% The IEEEtran \ifCLASSOPTIONcaptionsoff conditional can also be used
% later in the document, say, to conditionally put the References on a 
% page by themselves.




% *** PDF, URL AND HYPERLINK PACKAGES ***
%
\usepackage{url}
% url.sty was written by Donald Arseneau. It provides better support for
% handling and breaking URLs. url.sty is already installed on most LaTeX
% systems. The latest version and documentation can be obtained at:
% http://www.ctan.org/pkg/url
% Basically, \url{my_url_here}.




% *** Do not adjust lengths that control margins, column widths, etc. ***
% *** Do not use packages that alter fonts (such as pslatex).         ***
% There should be no need to do such things with IEEEtran.cls V1.6 and later.
% (Unless specifically asked to do so by the journal or conference you plan
% to submit to, of course. )


% correct bad hyphenation here
\hyphenation{op-tical net-works semi-conduc-tor}


\begin{document}
%
% paper title
% Titles are generally capitalized except for words such as a, an, and, as,
% at, but, by, for, in, nor, of, on, or, the, to and up, which are usually
% not capitalized unless they are the first or last word of the title.
% Linebreaks \\ can be used within to get better formatting as desired.
% Do not put math or special symbols in the title.
\title{\textbf{Improving Energy Efficiency and QoS performances Based on User State Predictions in Optical Access Networks}}
%\title{\Huge{\textbf{Dynamic Bandwidth and Wavelength Allocation Scheme for Next-Generation Wavelength-Aglie EPON}}}
%
%
% author names and IEEE memberships
% note positions of commas and nonbreaking spaces ( ~ ) LaTeX will not break
% a structure at a ~ so this keeps an author's name from being broken across
% two lines.
% use \thanks{} to gain access to the first footnote area
% a separate \thanks must be used for each paragraph as LaTeX2e's \thanks
% was not built to handle multiple paragraphs
%

\author{Jialong Li, Zhizhen Zhong, Nan Hua, Xiaoping Zheng, and Bingkun Zhou% <-this % stops a space
%\thanks{Manuscript received April 19, 2016; revised August 26, 2016; accepted August 26, 2016.}
%\thanks{M. Shell was with the Department
%of Electrical and Computer Engineering, Georgia Institute of Technology, Atlanta,
%GA, 30332 USA e-mail: (see http://www.michaelshell.org/contact.html).}% <-this % stops a space
\thanks{Manuscript received XX XX, 20XX; revised XX XX, 20XX; accepted XX XX, 20XX. This work was supported in part by projects under National 973 Program grant No. 2014CB340104/05 and NSFC under grant No. 61621064.}
\thanks{The authors are with the Tsinghua National Laboratory for Information Science and Technology (TNList), Department of Electronic Engineering, Tsinghua University, Beijing 100084, China (e-mail: xpzheng@mail.tsinghua.edu.cn; huan@mail.tsinghua.edu.cn).}}

% note the % following the last \IEEEmembership and also \thanks - 
% these prevent an unwanted space from occurring between the last author name
% and the end of the author line. i.e., if you had this:
% 
% \author{....lastname \thanks{...} \thanks{...} }
%                     ^------------^------------^----Do not want these spaces!
%
% a space would be appended to the last name and could cause every name on that
% line to be shifted left slightly. This is one of those "LaTeX things". For
% instance, "\textbf{A} \textbf{B}" will typeset as "A B" not "AB". To get
% "AB" then you have to do: "\textbf{A}\textbf{B}"
% \thanks is no different in this regard, so shield the last } of each \thanks
% that ends a line with a % and do not let a space in before the next \thanks.
% Spaces after \IEEEmembership other than the last one are OK (and needed) as
% you are supposed to have spaces between the names. For what it is worth,
% this is a minor point as most people would not even notice if the said evil
% space somehow managed to creep in.



% The paper headers
\markboth{Journal of Optical Communications and Networking}%
{Shell \MakeLowercase{\textit{et al.}}: Bare Demo of IEEEtran.cls for IEEE Journals}
% The only time the second header will appear is for the odd numbered pages
% after the title page when using the twoside option.
% 
% *** Note that you probably will NOT want to include the author's ***
% *** name in the headers of peer review papers.                   ***
% You can use \ifCLASSOPTIONpeerreview for conditional compilation here if
% you desire.




% If you want to put a publisher's ID mark on the page you can do it like
% this:
%\IEEEpubid{0000--0000/00\$00.00~\copyright~2015 IEEE}
% Remember, if you use this you must call \IEEEpubidadjcol in the second
% column for its text to clear the IEEEpubid mark.



% use for special paper notices
%\IEEEspecialpapernotice{(Invited Paper)}




% make the title area
\maketitle

% As a general rule, do not put math, special symbols or citations
% in the abstract or keywords.
\begin{abstract}
Optical access networks are widely implemented to provide higher bandwidth and lower end-to-end packet delay for subscribers. Among many access technologies, passive optical network (PON) is a priority for its energy efficiency feature as well as deployment scalability. With the rise of fiber to the home (FTTH), an Optical Network Unit (ONU) only serves a few subscribers and it will experience longer idle time than before, which leaves much room for energy efficiency. Due to the high energy consumption of the ONU, cyclic sleep mode is applied to reduce the energy consumption. The main idea of cyclic sleep mode is that the ONU goes into sleep status periodically when subscribers are in idle state. Conventional cyclic sleep schemes, which adopt static sleep time, strive to find a balance between energy efficiency and Quality of Service (QoS). In other words, the energy efficiency brought by cyclic schemes is with the cost of degradation of QoS, which may not satisfy the subscribers' requirements. Long sleep time is good for energy efficiency but may deteriorate the QoS, while short sleep time mitigates the degradation of QoS but impairs the energy efficiency. To improve the energy efficiency while maintain the QoS performances, in this paper, we classify the users' states into two types, i.e., active state and idle state. We adopt the long sleep time when users are idle to reduce energy consumption. If the users are in active state, we use the short sleep time to ensure the QoS. Since the sleep time need to be determined before an ONU goes into sleep status, we predict the users' states by their historical activities, which are from the actual data set. Simulation results show that our proposed scheme can achieve a considerable energy efficiency and QoS improvement.
\end{abstract}

% Note that keywords are not normally used for peerreview papers.
\begin{IEEEkeywords}
TDM-PON, Cyclic Sleep Mode, Energy Saving, QoS, FTTH.
\end{IEEEkeywords}






% For peer review papers, you can put extra information on the cover
% page as needed:
% \ifCLASSOPTIONpeerreview
% \begin{center} \bfseries EDICS Category: 3-BBND \end{center}
% \fi
%
% For peerreview papers, this IEEEtran command inserts a page break and
% creates the second title. It will be ignored for other modes.
\IEEEpeerreviewmaketitle



\section{Introduction}
% The very first letter is a 2 line initial drop letter followed
% by the rest of the first word in caps.
% 
% form to use if the first word consists of a single letter:
% \IEEEPARstart{A}{demo} file is ....
% 
% form to use if you need the single drop letter followed by
% normal text (unknown if ever used by the IEEE):
% \IEEEPARstart{A}{}demo file is ....
% 
% Some journals put the first two words in caps:
% \IEEEPARstart{T}{his demo} file is ....
% 
% Here we have the typical use of a "T" for an initial drop letter
% and "HIS" in caps to complete the first word.
\IEEEPARstart{F}{iber} to the home (FTTH) is being deployed worldwide to meet the huge bandwidth requirements driven by real-time video streaming and virtual reality (VR). According to the reports released by FTTH Council Europe, the number of FTTH subscribers in the world reaches 47 millions by the end of 2016. It is forecasted that 23 millions subscribers will be added in the next three years \cite{ftth}.

In general, we use FTTx to represent the fiber access networks. Several access architectures are classified in \cite{IEEFTTx}, including fiber to the curb (FTTC), fiber to the building (FTTB), and fiber to the desktop (FTTD). Compared to the FTTC/FTTB, fibers are terminated at users' home or even users' main computer in FTTH/FTTD, which can provide lower end-to-end packet delay and higher line rates. Among many access technologies, Passive Optical Network (PON) is widely implemented in FTTH for its passive features as well as easy deployment and scalability. International Telecommunication Union (ITU) has proposed the 10-Gigabit-capable symmetric passive optical network (XGS-PON) to accelerate the development of the access networks \cite{IEEERecommend}. XGS-PON is hopefully supporting higher data rates as well as staying backward compatible.

While more and more users enjoy the high-quality services provided by FTTH, the operators are concerned about the energy efficiency under the guideline of green telecommunication. It is estimated that up to 10\% of the world energy is used in Information and Communication Technology (ICT) industry while telecommunication plays a dominant role in energy consumption. Though PONs are more energy efficiency than copper-based access networks, they still are energy-hungry in the perspective of the operators. Among the components of PON, the Optical Network Units (ONUs) consume most of the energy, accounting for nearly 60\% of the whole system \cite{kani2013power}. In FTTH scenario, more ONUs will be deployed and energy effciency factor should be considered carefully.

Cyclic sleep mode has been defined in ITU standards to build an energy efficiency system. In cyclic sleep mode, there are two major states: active state and sleep state. In active state, ONUs can send and receive arriving packets, while transmitting and receiving functions are disabled in sleep state. ONUs are changed between the two states periodically to save energy and stay the ability to transmit data. Many previous works aiming at energy efficiency are based on cyclic sleep mode \cite{6375895, 5706311, 5360736, 6069715}. Mohammed et al. \cite{7489960} proposed an early wake-up algorithm, which aims at meeting the upstream packet delay while reducing energy consumption as much as possible. Dhaini et al. \cite{6512637} put forward a novel scheduling framework for green bandwidth allocation, which uses the batch-mode transmission feature to cut down the overhead brought by frequent power state transitions. Maneyama et al. \cite{6954287} presented a cyclic sleep control scheme to tackle with different Quality of Service (QoS) requirements. The work in \cite{kim2014dual} developed a dual cyclic sleep mode to control sleep time for the transmitter and receiver separately. Experiment results show that the proposed mode achieves better energy saving than the conventional cyclic mode. Nomura et al. \cite{nomura2012onu} studied energy efficiency performances by considering the users' traffic distributions. Numerical simulations indicate that up to 60\% energy consumption is reduced.

The reason why cyclic sleep mode can be applied to PON is that the channel is not always under heavy load and an idle ONU can be put into sleep state to save energy. The limitation of the aforementioned works is that the traffic generation is mainly based on FTTB scenario, where tens to hundreds of subscribers are connected to one ONU. In general, the aggregated traffic of an ONU is high at night in residential areas. On the contrary, the peak traffic flow usually happens in the daytime in business areas. This is called tidal effect and it is well studied in \cite{7444562, 6848766}. In FTTH scenario, only a few users are served by an ONU and the aggregated traffic can not be simply modeled by tidal effect. ONUs are idle in 80\% of the time, which leaves much room for energy efficiency. Further more, sleep period in cyclic sleep should be predefined before an ONU goes into sleep state. Improper sleep period may degrade the energy efficiency and influence the QoS performances. In this paper, we classify the user state into two types, namely idle and active state, and we adopt different sleep time when the users are in different states. When the users are in idle state, longer sleep period can be applied to improve the energy efficiency. If the users are in active state, we decrease the sleep period to ensure the QoS performances. To make our proposed scheme more practical in reality, the user states are predicted by their historical activities, which are from the actual dataset.

The rest of the paper is organized as follows. Section II presents the system model, including FTTH architecture based on PON, cyclic sleep mode, and energy consumption model. Section III illustrates the procedures and results of predicting user states based on different methods. In section IV, we show the numerical evaluations, and section V concludes the paper.


\section{System Model}

\begin{figure}[t]
    \centering 
        \includegraphics[width=1\columnwidth]{PON.pdf}\\ 
    \caption{ FTTH architecture based on PON.}
    \label{PON}
\end{figure}

\begin{figure}[t]
    \centering 
        \includegraphics[width=1\columnwidth]{sleep.pdf}\\ 
    \caption{ Cyclic sleep mode.}
    \label{sleep}
\end{figure}



In this section, we first introduce the FTTH architecture based on PON, then we illustrate the cyclic sleep mode, finally the energy consumption model is presented.

\subsection{FTTH Architecture Based on PON}
Fig. \ref{PON} depicts the FTTH architecture based on PON technology. A PON is a point-to-multipoint network with no active components between ONU and Opitcal Line Terminal (OLT), which locates in the Central Office (CO). In downstream, OLT is responsible for packet brocasting. Each ONU extracts its corresponding data and transfer to the corresponding users. In upstream, ONUs aggregate the packets and transmit them to the OLT. To avoid packet collisions, an ONU only transmits upstream data during the time slot assigned by OLT, which is called Time-Division Multiple Access (TDMA). The interior elements between the OLT and ONU are power splitters. Since fewer subscribers are supported by one ONU in FTTH scenarios, more ONUs are required in a single PON system. Considering the power budget, a typical split ratio ranges from 32 to 256. Higher split ratio can be achieved by adding more wavelengths in the system. The authors in \cite{luo2013time} developed a PON prototype with a 1 : 512 split ratio.

\subsection{Dynamic Bandwidth Allocation Scheme}
As we mention before, the OLT grants the access to each ONU in upstream channel, and it needs an efficient way to assign the time slot, which it is called Dynamic Bandwidth Allocation (DBA). In this paper, we adopt interleaved polling with adaptive cycle time (IPACT) \cite{983911}, which is a widely used DBA scheme. The procedures of IPACT are described blow. Before the $k$th ONU transmitting upstream data, the OLT sends a control message to it. The control message, called $Grant$ here, contains the time slot size information. Upon receiving the $Grant$, the $k$th ONU starts sending its packets during the granted time slot. At the end of the transmission, the $k$th ONU generates its own control message ($Report$) information, reporting how many data are left in the buffer. In the next polling cycle, the OLT will grant time slot size according to the $Report$. Due to the propagation delay between the OLT and ONU (typically 1 $\sim$ 100 $\mu$s), to make full use of channel bandwidth, the OLT sends $Grant$ to the $(k + 1)$th ONU before receiving packets from the $k$th ONU. Since the OLT knows the Round-Trip Time (RTT) to each ONU in the very beginning, it knows when it receives the last bit of the $k$th ONU. So the OLT sends $Grant$ in a proper time to make sure that the first bit of the $(k + 1)$th ONU will arrive soon after the last bit of the $k$th ONU arrives. Usually we set a small guard time interval ($T_{guard}$) to ensure safety.

Note that, if the OLT grants time slot as much as the ONU requires, a heavy-load ONU may monopolize the channel for a long time, then packets from other ONUs may experience an excessive delay. To avoid this problem, we limit the time slot size to an upper bound, denoted $Max$\textunderscore $Grant$ here. The $Grant$ generated by the OLT can be expressed as follows.

\begin{footnotesize}
\begin{equation}
Grant  =  
\left\{
\begin{aligned}
& Report\textunderscore data, \quad if \ Report\textunderscore data <= Max\textunderscore Grant & \\
& Max\textunderscore Grant, \quad otherwise & \\
\end{aligned}
\right.
\end{equation}
\end{footnotesize}



\subsection{Cyclic Sleep Mode}

\begin{figure}[t]
    \centering 
        \includegraphics[width=0.9\columnwidth]{status.pdf}\\ 
    \caption{ State transfer diagram.}
    \label{status}
\end{figure}


Fig. \ref{sleep} shows the cyclic sleep mode. The main idea of cyclic sleep mode is that when no packets arrive, the ONU switches to the sleep state to reduce energy consumption. There are four states in a completed sleep cycle.

\textbf{Active}: In active status, both ONU receivers and transmitters remain on. The ONU stays in high power level and all functions are enabled.

\textbf{Awaiting}: A transition status before an ONU goes into sleep status. Awaiting status can avoid too frequent transitions between active and sleep status. Like active status, receiver and transmitter are working normally in awaiting status. During the awaiting status, when the ONU receives packets, it can switch to active status to send data. If no packet arrives or the buffer is empty during the awaiting time ($T_{awaiting}$), the ONU will turn to sleep status.

\textbf{Sleep}: Both receivers and transmitters are powered off, staying in a lower power level. During the sleep status, the OLT stops sending control message to the ONU. The OLT needs to determine the sleep time ($T_{sleep}$) before an ONU goes into sleep status. Incoming packets have to be buffered and only can be transmitted in the active state.

\textbf{Recovery}: After the sleep time expires, the ONU needs to power up its receivers and transmitters, synchronize with the OLT, which is called recovery status. The recovery time ($T_{recovery}$) ranges from 2ms to 5ms \cite{mandin2008epon}.

Fig. \ref{status} is the state transfer diagram, showing the relations between four status. In the following, we present the procedures of a completed sleep cycle.

\textbf{1}. In a polling cycle, an ONU has no packets to transmit and generates the $Report$ message, telling the OLT that its buffer is empty. Then the ONU goes into awaiting status. The ONU 
can quit the awaiting status in advance if packets arrive during the awaiting time, and report to the OLT how many data it want to transfer in the next polling cycle.

\textbf{2}. After $T_{awaiting}$ time, the ONU's buffer is still empty, and it will ask the OLT for sleep. Then the OLT instructs the ONU to sleep status. When the ONU is in sleep status, the OLT will skip the sleeping ONU until the sleep time expires.

\textbf{3}. At the end of the sleep time, the ONU needs to power up itself and synchronize with the OLT. After $T_{recovery}$ time, it sends message to the OLT, reporting how many data are in the buffer. 

\textbf{4}. If the ONU has required the time slot, it can turn into active status and transmit packets in the next polling cycle. If the reporting data is zero, then the ONU goes into awaiting status and are ready for the next sleep cycle.

\subsection{Power Consumption Model}
In this part, we mainly focus the energy consumption in the ONU side. Once we know the time that an ONU stays in each status, we could calculate the power consumption. Active, awaiting, and recovery status are in high power level\footnote{In fact, the power consumption of recovery status is between high and low power level. For simplicity we assume the power comsumption is the same as active status. Readers could refer to \cite{6172273} for more details.}, and we use $P_{high}$ to represent the power consumption. In sleep status, the power consumption is $P_{low}$. And the average power consumption for an ONU can be calculated as follows.

\begin{footnotesize}
\begin{equation}
P = \frac{\sum({T_{active} + T_{awaiting} + T_{recovery}}){P_{high}} + \sum T_{sleep}P_{low}}{\sum( {T_{active} +  T_{awaiting} + T_{recovery} + T_{sleep}})}
\end{equation}
\end{footnotesize}

From the formulation, we could learn that long sleep time contributes to reduce the energy consumption. Short sleep time will conduct frequent power states transitions, and it will harm the energy efficiency since energy is wasted in frequent recovery processes. So in the perspective of energy efficiency, long sleep time would be a good choice. However, long sleep time may deteriorate the QoS performances, such as packet delay, packet loss ratio. The sleep time should be determined carefully to stay energy efficieny while satisfy the users' requirements. 

\section{Modeling on User States}

In previous section, we know that cyclic sleep mode will activate when the ONU is idle. In FTTH/FTTD scenarios, an ONU only serves a few subscribers, and the ONU status could be inferred from the corresponding users' states. To make our model more close to the real situations, the user states are from the actual dataset. In this section, we try to model the ONU's status by knowing the users' states in FTTH/FTTD scenarios.

\subsection{Dataset description}
To depict the user state distributions more precisely, we model the user states by actual dataset \cite{Datatang}. This dataset records 1000 Internet users' activity logs in four weeks, through which we could know when the users are using Internet service. For simplicity, we classify the user states into two types, namely active and idle state. We divide a day into 24 periods, in other words, each period last for an hour. During the period, if the user uses Internet service, we call this user active, otehrwise idle. According to the activity logs, we could infer the user states at each period. Further more, the dataset contains some demographic information about the 1000 Internet users, including gender, date of birth, and occupation, etc. Note that the users' names are not known for privacy protection reason. 

Fig. \ref{user_state} shows four users' states in a week, who are selected randomly from the 1000 Internet users. Fig. \ref{user_state} indicates that the users' active time may be different from each other. In addition, an ONU only support several subscribers, and the traffic load can not be simply modeled by tidal effect. Further more, we can observe that the users are idle in most of the time. Take the upper right user as an example, whose active time is the least among the four users, the active time of this user only accounts for 10\% of the whole time. In fact, according to the statistics of the 1000 Internet users, the active and idle time are 20\% and 80\%, respectively. Under the guideline of green optical access networks, there should be a high efficient algorithm for energy saving.

If we know the user states, we could infer the status of the ONU that connects to these users. When the ONUs are idle, cyclic sleep mode are activated and longer sleep period could be applied. In that case, more energy are saved and it does not degrade the QoS since no users are using Internet service. The question is in cyclic sleep mode, we need to determine the sleep time before an ONU goes into sleep status, which means the OLT are required to estimate the ONU's status before it grants the sleep period. Luckily, from the graph, we can realize that the users' active and idle time are regular in some degree. In next subsection, we introduce two methods on how to predict the user states.

\subsection{How to Predict the User States}

\begin{figure}[!t]
  \centering 
  \includegraphics[width=1\columnwidth]{user_state.pdf}\\ 
  \caption{ Four users' states in a week.}
  \label{user_state}
\end{figure}

\begin{figure}[!t]
  \centering 
  \includegraphics[width=1\columnwidth]{tree.pdf}\\ 
  \caption{ A simple case for decision tree.}
  \label{tree}
\end{figure}

The first method is a simple one. Intuitively, the users' active and idle states occur periodically. For users with different characteristics, like occupation, age, and gender, they may have different periodic features. So a straightforward way to estimate a user's state is to use his/her state logs in last several weeks. For example, if a user's states are mostly active in last several Friday nights, it is very likely that this user is active in next Friday night. Fig. \ref{user_state} supports our assumption in some degree. The dataset we mention before provides activity logs in four weeks. To evaluate the prediction methods' performances, we divide the dataset into two parts, i.e., train set and test set. Train set contains activity logs in the first three weeks and test set contains the last week.

So the first method in estimating the user states, denoted Lweek in this paper, is expressed below. To estimate a user's state in a certain period, we count the most occured state in that period in the last three weeks and use it to represent the predicted result. This method does not require complex algorithms, but lacks of accuracy, which we will see in the next subsection.

The second method is called decision tree (Dtree) \cite{Kotsiantis2013}, which is used for classification in machine learning. Our target is to decide a user's state whether it is active or idle, and that could be regarded as a classification problem. Fig. \ref{tree} depicts a simple case for decision tree. The decision tree is built according to the users' historical logs with the users' features we selected, which are date, age, time, and gender in this simple case. The leaf nodes in the decision tree act as the classification results, i.e., active or idle state in our task. The non-leaf nodes serve as the classification conditions. Upon the decison tree is constructed, we could use it to predict the user states. Take Fig. \ref{tree} as an example, we want to know a middle-age male user's state in Sunday night. Look the picture from top to bottom, in the first calssification condition, we select "day off" side. Then in the age classification condition, we choose the "middle" branch. And in the final classification condition, according to the user's gender, male, we could estimate that this user's state is active in Sunday night. The path is marked by red arrows in Fig. \ref{tree}. Note that the diagram does not represent the truth of our dataset. In fact, the decision tree constructed by the train set is larger and much more complex than this one. 

In this paragraph, we discuss how we build the decision tree using the train set. As we stated before, we divide a day into 24 periods and a user has 24 states in a day. Then a state can be written as the following format: $(user$\textunderscore $id$, $gender, age, education, occupation, time, date)$. This is a completed sample of a state. The elements of a sample are called features. In our task, we select seven features to build a sample. The description of the features are discussed below.

\begin{itemize}
  \item[-] $user$\textunderscore $id$: The 1000 Internet users are marked with from 0 to 999.
  \item[-] $gender$: Male or female.
  \item[-] $age$: Three categories. Young, middle-age, and old.
  \item[-] $education$: Three categories. Under middle school, senior high school, undergraduate or above.
  \item[-] $occupation$: Eleven categories, including students, peasants, company managers, etc.
  \item[-] $time$: The period a state belongs to. Marked with from 0 to 23.
  \item[-] $date$: From Monday to Sunday.
\end{itemize}

In train set, we have 504000 samples (1000 users $\times$ 3 weeks $\times$ 7 days $\times$ 24 periods) and 168000 samples in test set. After the construction of decision tree using train set, we use the samples in test set to estimate a user's state by the tree. The test set is used for checking out the performance of the tree. 

Here we analyze the complexity in generating a decision tree. We use $M$ and $N$ to represent the number of samples and features, respectively. There are several criterions in constructing a tree, and the shape of the tree may be diverse if we apply different criterions. The prediction results of decision tree with different shape may have a slight difference. Here we adopt Classification and Regression Trees (CART). CART constructs binary trees that yields the largest information gain at each node. In general, the complexity of constructing a balanced binary trees is $O(Mlog(M))$. Though CART attempts to build balanced binary trees, the tree will not always be strictly balanced. We assumes that the subtrees of the decision tree are approximately balanced. Each node has to find the feature that provides the largest reduction in entropy, whose complexity is $O(N)$. There are $M$ nodes and the total complexity is $O(M^{2}Nlog(M)$.

\subsection{Prediction Results}
When the decision tree is completed, we could use the samples in test set to evaluate the prediction performances. In this part, we show the results by two prediction methods, Lweek and Dtree.

\begin{table}[H]
\centering
\caption{Prediction Results in FTTD Scenario}
\label{table_FTTD}
\begin{tabular}{llrrc}
  \toprule
Ground Truth & Precition & Dtree & Lweek & Type\\  
  \midrule
active & active & 10.05\% & 7.80\% & True\\  
active & idle & 8.09\% & 10.34\% & False\\ 
idle & active & 5.28\% & 10.69\% & False\\
idle & idle & 76.58\% & 71.17\% & True\\ 
  \midrule
\qquad \qquad Total Accuracy &  & 86.63\% & 78.97\%\\ 
  \bottomrule
\end{tabular}  
\end{table}  

\begin{table}[H]
\centering
\caption{Prediction Results in FTTH Scenario}
\label{table_FTTH}
\begin{tabular}{llrrc}
  \toprule
Ground Truth & Precition & Dtree & Lweek & Type\\  
  \midrule
active & active & 16.92\% & 12.75\% & True\\  
active & idle & 12.34\% & 16.52\% & False\\ 
idle & active & 7.88\% & 18.40\% & False\\
idle & idle & 62.86\% & 52.33\% & True\\ 
  \midrule
\qquad \qquad Total Accuracy &  & 79.78\% & 65.08\%\\ 
  \bottomrule
\end{tabular}  
\end{table}

\subsubsection{FTTD Scenario} First we show the results in FTTD scenario. The decision tree is used for the user states estimation, and we need to infer the ONU's status according to the states of the users it connects to. In FTTD scenario, an ONU only serves a subscriber, so their status remain the same.

Table \ref{table_FTTD} demonstrates the prediction results in FTTD scenario. Ground truth means the actual state of the user and prediction means the predicted states by Dtree or Lweek, two methods on estimating the user states that we have showed before. True or False stands for whether we predict a state accurately or not. Dtree predicts 15.33\% active states and 10.05\% of over all samples are accurate while 5.28\% are inaccurate. 85.67\% idle states are predicted and 76.58\% of overall samples are accurate while 8.09\% are inaccurate. The total accuracy of Dtree is 86.63\%. For Lweek, it predicts 18.49\% active states and 7.80\% of over all samples are accurate while 10.69\% are inaccurate. 81.51\% idle states are predicted and 71.17\% of overall samples are accurate while 10.34\% are inaccurate. The total accuracy of Lweek is 78.97\%. According to the toal accuracy, we know that Dtree predicts more accurately than Lweek.

From Table \ref{table_FTTD}, we find another interesting fact. The actual active states accounts for 18.14\% and idle states are 81.86\%. In the predicted results by Lweek, the active and idle states make up 18.49\% and 81.51\%, respectively. The proportions are pretty close, though the accuracy of Lweek is not very ideal. It is because the proportions of active and idle states for a large amount of users are stable at a period of time. Lweek predicts user states directly by historical states, its predicted results portray the characteristics of the user states in the past time. As for Dtree, it uses more features and builds decision tree based on maximum information gain, which performs better than Lweek.

\subsubsection{FTTH Scenario}
In FTTH scenario, an ONU serves several users and predicting an ONU's status becomes a littel more complex. We suppose the number of ONUs is 128 and select 256 users randomly. Then we assign the users to the ONUs in the following rules. The number of users assigned to each ONU ranges from one to three. In this paper, we assume that if one or more than one users are active, then the associated ONU's status is active. Only all connected users are idle can we call an ONU idle. 

Table \ref{table_FTTH} demonstrates prediction results in FTTH scenario. Dtree predicts 24.80\% active states and 16.92\% of over all samples are accurate while 7.88\% are inaccurate. 75.20\% idle states are predicted and 62.86\% of overall samples are accurate while 12.34\% are inaccurate. The total accuracy of Dtree is 79.78\%. For Lweek, it predicts 31.15\% active states and 12.75\% of over all samples are accurate while 18.40\% are inaccurate. 68.85\% idle states are predicted and 52.33\% of overall samples are accurate while 16.52\% are inaccurate. The total accuracy of Lweek is 65.08\%. From the results we can learn that the accuracy rates decrease compared to FTTD scenario.

\section{Performance Evaluation}
\subsection{Simulation Setup}

\begin{table}[t]
\centering
\caption{Simulation Parameters}
\label{parameter}
\begin{tabular}{llc}
  \toprule
Symbol & Meaning & Value\\  
  \midrule
$N$ & The number of ONUs & 128\\  
$RTT$ & Round-trip time & 0.2ms\\ 
$T_{sleep}$ & sleep time in a sleep cycle & 5-100ms\\
$T_{awaiting}$ & \tabincell{c}{Awaiting time before an ONU\\goes into sleep status} & 5ms\\
$T_{recovery}$ & \tabincell{c}{Recovery and synchronization time \\needed to power up an ONU} & 2ms\\ 
$T_{guard}$ & \tabincell{c}{Interval between upstream data \\ from two ONUs} & 5$\mu$s\\ 
$P_{high}$ & energy consumption in active state & 6.35W\\ 
$P_{low}$ & energy consumption in idle state  & 0.70W\\
$\lambda$ & packet arrive rate & 500-1500 frames/s\\ 
$B$ & Buffer size & 1M\\
$Max$\textunderscore${Grant}$ & \tabincell{c}{Maximum time slot assigned to \\an ONU in a polling cycle} & 50$\mu$s\\ 
$R_{u}$ & Upstream channel bandwidth & 1G/s\\ 
  \bottomrule
\end{tabular}  
\end{table}

In this section, we evaluate the performances of our proposed scheme on FTTD and FTTH scenario, respectively. Besides energy efficiency, we investigate two main factors associated to QoS, packey delay and packet loss ratio, in this paper. Simulation parameters are showed on Table \ref{parameter}, whihc are mainly from \cite{6172273}. We compare our scheme to the conventional cyclic sleep policy, which adopts static sleep time and is denoted standard method in this paper.

\subsection{FTTD Scenario}

\begin{figure*}[!t]
    \centering 
    \subfigure[] { \label{fig:a} 
        \includegraphics[width=0.63\columnwidth]{128_standard_energy.pdf} 
    } 
    \subfigure[] { \label{fig:b} 
        \includegraphics[width=0.63\columnwidth]{128_standard_delay.pdf} 
    }
    \subfigure[] { \label{fig:c} 
        \includegraphics[width=0.63\columnwidth]{128_standard_loss.pdf} 
    } 
    \caption{Comparison between energy consumption (a), device lifetime (b), and migrated traffic amount (c) with a different number of postponed working wavelengths and postponed periods under traffic-III.} 
    \label{standard} 
\end{figure*}

On FTTD scenario, we randomly select 128 users and assign to 128 ONUs in order. The packet arrival process follows Poisson distribution and the packet size is selectly from 64 to 1518 bytes uniformly. The packet arrive rate for a active user is 1000 frames per second, while an idle user only generates 10 frames per second. 

Fig. depicts the energy consumption, packet delay, and packet loss performances with static sleep time, namely the standard method. The red line, labeled standard in the figure, is composed with 20 points. Each point means we adopt static sleep time, which are 5, 10, 15, ... , 100ms, respectively. It is obvious that with the raise of sleep time, energy consumption declines. From Fig. (a), the average energy consumption for each ONU when adopted cyclic sleep mode is less than 6.35W, which is the energy comsumption when a ONU is always in active status. We could learn that the cyclic sleep mode could achieve significant energy savings since users are idle in most of the time. The cost for energy efficiency is the deterioration of packet delay and packet loss ratio, which are shown in Fig (b) and (c). The packet delay increases with the raise of sleep time. The packet loss ratio is zero when the sleep time is less than 60ms, but the ratio will increase sharply if we prolong the sleep time. When the sleep time is 100ms, the packet loss ratio is almost 5\%.

\begin{figure*}[!t]
    \centering 
    \subfigure[] { \label{fig:a} 
        \includegraphics[width=0.63\columnwidth]{128_predict_energy.pdf} 
    } 
    \subfigure[] { \label{fig:b} 
        \includegraphics[width=0.60\columnwidth]{128_predict_delay.pdf}
    }
    \subfigure[] { \label{fig:c} 
        \includegraphics[width=0.63\columnwidth]{128_predict_loss.pdf} 
    } 
    \caption{Comparison between energy consumption (a), device lifetime (b), and migrated traffic amount (c) with a different number of postponed working wavelengths and postponed periods under traffic-III.} 
    \label{predict} 
\end{figure*}

Fig. shows the simulation results when we apply two kinds of sleep time, i.e., short and long sleep time, according to the prediction results of ONU status. The short sleep time in the Fig. is fixed to 10ms while long sleep time ranges from 15ms to 100ms. The green line, marked precise in Fig., means the prediction results are 100\% correctly. In other words, we know an ONU's status in advance. Similar to the performances by standard method, energy efficiency is improved when long sleep time increase with the degeneration of packet delay and packet loss ratio. From the pictures, we could learn that precise method could achieve high energy efficiency with slight QoS degeneration. For packet delay performances, the delay is around 1ms when long sleep time is 15ms and the delay is only about 3ms if we increase the sleep long time to 100ms. For packet loss ratio performances, precise method realizes zero packet loss for all long sleep time. It is because we apply short sleep period to prevent packet loss when the ONU is in active status. Long sleep period is implemented in idle status and does not bring about packet loss since the packet arrive rate is extremely low. 

Though predicting an ONU's status 100\% precisely is hardly possible, we still could improve the prediction accuracy by some classification algorithms. In the previous section, we have achieved 86.63\% accuracy rate by Dtree and 78.97\% by Lweek. And from Fig., performances by Dtree are better than Lweek since the accuracy rate of Dtree is higher than Dtree. When accuracy rate is 100\%, it is the precise line.

\begin{figure*}[!t]
    \centering 
    \subfigure[] { \label{fig:a} 
        \includegraphics[width=0.63\columnwidth]{128_segment_delay_1.pdf} 
    } 
    \subfigure[] { \label{fig:b} 
        \includegraphics[width=0.63\columnwidth]{128_segment_delay_2.pdf} 
    }
    \subfigure[] { \label{fig:c} 
        \includegraphics[width=0.63\columnwidth]{128_segment_delay_3.pdf} 
    }
    \subfigure[] { \label{fig:d} 
        \includegraphics[width=0.63\columnwidth]{128_segment_loss_1.pdf} 
    } 
    \subfigure[] { \label{fig:e} 
        \includegraphics[width=0.63\columnwidth]{128_segment_loss_2.pdf} 
    }
    \subfigure[] { \label{fig:f} 
        \includegraphics[width=0.63\columnwidth]{128_segment_loss_3.pdf} 
    }
    \caption{Comparison between energy consumption (a), device lifetime (b), and migrated traffic amount (c) with a different number of postponed working wavelengths and postponed periods under traffic-III.} 
    \label{segment} 
\end{figure*}

Next, we compare the performances by standard method with three prediction methods, namely Dtree, Lweek, and precise, respetively. To make the differences more clear, we depict the energy-delay and energy-packet loss graphs. In this part, we fix the short sleep time to 10ms, 20ms, 40ms, and 60ms. The associated long sleep time is larger than the short sleep time but no more than 100ms. For each short sleep time, we select ten different values for the long sleep time. So there are total forty points in the graph for each prediction method. 

Fig. (a-c) show energy-delay performance comparision between standard method and prediction methods, and Fig. (a-c) show energy-packet loss performances. Points for each prediction method are divided into two parts, under and upper the standard line. It is clear that points under the standard line have a better performance than standard method in both energy efficiency and QoS. On the contrary, the performances of upper points is worse than the standard method. The gap between the under points and the standard line is larger, the more performance improvement is achieved. In Fig. (c), all points are under the standard line. As for the Dtree and Lweek method, two points are upper the standard line in Fig. (b) and more than ten upper points in Fig. (c). It is because compared to the precise method, Dtree and Lweek could not predict an ONU's status 100\% accurately. When an active ONU is predicted as idle status, long sleep time will be applied instead of short sleep time, which leads to longer packet delay and higher packet loss ratio, though more energy consumptions are saved. If an idle ONU is estimated as active status, the influence on packet delay and packet loss ratio is negligible, but energy efficiency deteriorates since short sleep time is adopted. The inaccurate prediction results will compensate the energy efficiency and QoS improvement by predition method. That is why some points' performances are worse than the standard method.

It could be noticed that in Fig. (d-e), compared to the energy-delay performances, more points are upper the standard line. It is because packet loss ratio is more sensitive to the long sleep time. For example, we adopt 10ms as the short sleep period and 100ms as the long sleep period. When we incorrectly predict an active ONU as idle status, we apply 100ms as the sleep time as it occur packet loss. In standard method, to achieve the same energy efficiency, the cyclic sleep time is between the short and long sleep time, say 20ms. From Fig. (c), we could learn that 20ms could not occur packet loss. In this case, the packet loss ratio of prediction method is larger than the standard method. Under the accuracy of prediciton results we have made, we suggest the difference between short and long sleep time should not too large, aiming at avoiding packet loss performance degeneration. In that case, we still could achieve considerable energy efficiency and QoS improvement compared to standard method.

\subsection{FTTH Scenario}

\begin{figure*}[!t]
    \centering 
    \subfigure[] { \label{fig:a} 
        \includegraphics[width=0.63\columnwidth]{256_segment_delay_1.pdf} 
    } 
    \subfigure[] { \label{fig:b} 
        \includegraphics[width=0.63\columnwidth]{256_segment_delay_2.pdf} 
    }
    \subfigure[] { \label{fig:c} 
        \includegraphics[width=0.63\columnwidth]{256_segment_delay_3.pdf} 
    }
    \subfigure[] { \label{fig:d} 
        \includegraphics[width=0.63\columnwidth]{256_segment_loss_1.pdf} 
    } 
    \subfigure[] { \label{fig:e} 
        \includegraphics[width=0.63\columnwidth]{256_segment_loss_2.pdf} 
    }
    \subfigure[] { \label{fig:f} 
        \includegraphics[width=0.63\columnwidth]{256_segment_loss_3.pdf} 
    }
    \caption{Comparison between energy consumption (a), device lifetime (b), and migrated traffic amount (c) with a different number of postponed working wavelengths and postponed periods under traffic-III.} 
    \label{segment} 
\end{figure*}

On FTTH scenario, we show the energy efficiency and QoS improvement by predicition method. Compared to FTTD scenario, we could notice that more points are upper the standard line and the under points are much closer to the standard line. In other words, the energy efficiency and QoS improvement is not as good as FTTD's. It is because the prediction accuracy on FTTH scenario is lower than on FTTD's. Dtree and Lweek realize 86.63\% and 78.97\% on FTTD scenario, respective. On FTTH scenario, these numbers drop to 79.78\% and 65.08\%, respectively. In Fig. , we could learn that through Lweek method, only a small part of the points could achieve slight improvement on energy efficiency and QoS. But for Lweek, shown in Fig. , quite a few points could realize considerable energy efficiency and QoS improvement. 

\section{Conclusion}
In this paper, we study the energy saving and QoS performances by cyclic sleep mode in optical access networks. Conventional cyclic sleep mode, which adopts static sleep time, strives to find a balance between energy saving and QoS performances. To further improve the energy efficiency and QoS performances in the optical access networks, we propose a novel cyclic sleep mode that apply long sleep time in idle status while short sleep time in active status. Since the sleep time needed to be predefined, we propose two prediction methods, Lweek and Dtree, to estimate the ONUs' status. Simulation results show that Dtree and Lweek could improve considerable energy efficiency and QoS performances improvement compared to conventional cyclic sleep mode. Since the accuracy of prediction results of Dtree is better than Lweek's, Dtree could provide better performances than Lweek, and the complexity of Dtree is acceptable. In future work, we try to upgrade the accuracy of prediction results, which could further improve the energy efficiency and QoS performances.


\section*{Acknowledgment}
This work was supported in part by projects under National 973 Program grant No. 2014CB340104/05 and NSFC under grant No. 61621064.

%\subsection{Subsection Heading Here}
%Subsection text here.

% needed in second column of first page if using \IEEEpubid
%\IEEEpubidadjcol

%\subsubsection{Subsubsection Heading Here}
%Subsubsection text here.


% An example of a floating figure using the graphicx package.
% Note that \label must occur AFTER (or within) \caption.
% For figures, \caption should occur after the \includegraphics.
% Note that IEEEtran v1.7 and later has special internal code that
% is designed to preserve the operation of \label within \caption
% even when the captionsoff option is in effect. However, because
% of issues like this, it may be the safest practice to put all your
% \label just after \caption rather than within \caption{}.
%
% Reminder: the "draftcls" or "draftclsnofoot", not "draft", class
% option should be used if it is desired that the figures are to be
% displayed while in draft mode.
%
%\begin{figure}[!t]
%\centering
%\includegraphics[width=2.5in]{myfigure}
% where an .eps filename suffix will be assumed under latex, 
% and a .pdf suffix will be assumed for pdflatex; or what has been declared
% via \DeclareGraphicsExtensions.
%\caption{Simulation results for the network.}
%\label{fig_sim}
%\end{figure}

% Note that the IEEE typically puts floats only at the top, even when this
% results in a large percentage of a column being occupied by floats.


% An example of a double column floating figure using two subfigures.
% (The subfig.sty package must be loaded for this to work.)
% The subfigure \label commands are set within each subfloat command,
% and the \label for the overall figure must come after \caption.
% \hfil is used as a separator to get equal spacing.
% Watch out that the combined width of all the subfigures on a 
% line do not exceed the text width or a line break will occur.
%
%\begin{figure*}[!t]
%\centering
%\subfloat[Case I]{\includegraphics[width=2.5in]{box}%
%\label{fig_first_case}}
%\hfil
%\subfloat[Case II]{\includegraphics[width=2.5in]{box}%
%\label{fig_second_case}}
%\caption{Simulation results for the network.}
%\label{fig_sim}
%\end{figure*}
%
% Note that often IEEE papers with subfigures do not employ subfigure
% captions (using the optional argument to \subfloat[]), but instead will
% reference/describe all of them (a), (b), etc., within the main caption.
% Be aware that for subfig.sty to generate the (a), (b), etc., subfigure
% labels, the optional argument to \subfloat must be present. If a
% subcaption is not desired, just leave its contents blank,
% e.g., \subfloat[].


% An example of a floating table. Note that, for IEEE style tables, the
% \caption command should come BEFORE the table and, given that table
% captions serve much like titles, are usually capitalized except for words
% such as a, an, and, as, at, but, by, for, in, nor, of, on, or, the, to
% and up, which are usually not capitalized unless they are the first or
% last word of the caption. Table text will default to \footnotesize as
% the IEEE normally uses this smaller font for tables.
% The \label must come after \caption as always.
%
%\begin{table}[!t]
%% increase table row spacing, adjust to taste
%\renewcommand{\arraystretch}{1.3}
% if using array.sty, it might be a good idea to tweak the value of
% \extrarowheight as needed to properly center the text within the cells
%\caption{An Example of a Table}
%\label{table_example}
%\centering
%% Some packages, such as MDW tools, offer better commands for making tables
%% than the plain LaTeX2e tabular which is used here.
%\begin{tabular}{|c||c|}
%\hline
%One & Two\\
%\hline
%Three & Four\\
%\hline
%\end{tabular}
%\end{table}


% Note that the IEEE does not put floats in the very first column
% - or typically anywhere on the first page for that matter. Also,
% in-text middle ("here") positioning is typically not used, but it
% is allowed and encouraged for Computer Society conferences (but
% not Computer Society journals). Most IEEE journals/conferences use
% top floats exclusively. 
% Note that, LaTeX2e, unlike IEEE journals/conferences, places
% footnotes above bottom floats. This can be corrected via the
% \fnbelowfloat command of the stfloats package.


% if have a single appendix:
%\appendix[Proof of the Zonklar Equations]
% or
%\appendix  % for no appendix heading
% do not use \section anymore after \appendix, only \section*
% is possibly needed

% use appendices with more than one appendix
% then use \section to start each appendix
% you must declare a \section before using any
% \subsection or using \label (\appendices by itself
% starts a section numbered zero.)
%




% you can choose not to have a title for an appendix
% if you want by leaving the argument blank



% use section* for acknowledgment



% Can use something like this to put references on a page
% by themselves when using endfloat and the captionsoff option.
\ifCLASSOPTIONcaptionsoff
  \newpage
\fi



% trigger a \newpage just before the given reference
% number - used to balance the columns on the last page
% adjust value as needed - may need to be readjusted if
% the document is modified later
%\IEEEtriggeratref{8}
% The "triggered" command can be changed if desired:
%\IEEEtriggercmd{\enlargethispage{-5in}}

% references section

% can use a bibliography generated by BibTeX as a .bbl file
% BibTeX documentation can be easily obtained at:
% http://mirror.ctan.org/biblio/bibtex/contrib/doc/
% The IEEEtran BibTeX style support page is at:
% http://www.michaelshell.org/tex/ieeetran/bibtex/
%\bibliographystyle{IEEEtran}
% argument is your BibTeX string definitions and bibliography database(s)
%\bibliography{IEEEabrv,../bib/paper}
%
% <OR> manually copy in the resultant .bbl file
% set second argument of \begin to the number of references
% (used to reserve space for the reference number labels box)
%\begin{thebibliography}{1}

%\bibitem{IEEEhowto:kopka}
%H.~Kopka and P.~W. Daly, \emph{A Guide to \LaTeX}, 3rd~ed.\hskip 1em plus
%  0.5em minus 0.4em\relax Harlow, England: Addison-Wesley, 1999.

%\end{thebibliography}

% biography section
% 
% If you have an EPS/PDF photo (graphicx package needed) extra braces are
% needed around the contents of the optional argument to biography to prevent
% the LaTeX parser from getting confused when it sees the complicated
% \includegraphics command within an optional argument. (You could create
% your own custom macro containing the \includegraphics command to make things
% simpler here.)
%\begin{IEEEbiography}[{\includegraphics[width=1in,height=1.25in,clip,keepaspectratio]{mshell}}]{Michael Shell}
% or if you just want to reserve a space for a photo:

%\begin{IEEEbiography}{Michael Shell}
%Biography text here.
%\end{IEEEbiography}

% if you will not have a photo at all:
%\begin{IEEEbiographynophoto}{John Doe}
%Biography text here.
%\end{IEEEbiographynophoto}

% insert where needed to balance the two columns on the last page with
% biographies
%\newpage

%\begin{IEEEbiographynophoto}{Jane Doe}
%Biography text here.
%\end{IEEEbiographynophoto}

% You can push biographies down or up by placing
% a \vfill before or after them. The appropriate
% use of \vfill depends on what kind of text is
% on the last page and whether or not the columns
% are being equalized.

%\vfill

% Can be used to pull up biographies so that the bottom of the last one
% is flush with the other column.
%\enlargethispage{-5in}

\bibliographystyle{IEEEtran}%
\bibliography{jocn2017}

% that's all folks
\end{document}


